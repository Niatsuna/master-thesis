\chapter{State-of-the-Art Simulator Analysis}
A comprehensive analysis of the current edge computing simulation landscape is the focus of the following sections.
The analysis provides the foundation for developing a systematic evaluation framework for edge computing simulators and begins by establishing
clear selection criteria for identifying representative simulators from the extensive body of research and development in this domain.
Following a systematic selection process, ten representative edge computing simulators are examined, each offering distinct approaches to modeling edge computing environments.
The chapter proceeds with detailed simulator profiles that introduce their core architectures, capabilities and inteded use cases, providing readers with essential background for understanding their respective strength and limitations.
To support the systematic comparison of the ten representative simulators, we establish a analysis framework which established aspects and categories in which the comparison should take place.
Subsequently, a structured comparative analysis examines these simulators in comparison to their competitors which reveals significant patterns and variations within the simulation landscape.
These significant key aspects are highlighted and missing aspects are covered in a comprehensive gap analysis, establishing the foundation for the requirements catalogue developed in the subsequent chapter.

\section{Selection Criteria and Methodology}
\todo{Academic relevance, technical diversity, trends, application scope}
\section{Representative Simulator Overview}
\todo{Quick introduction to all simulators}
\section{Comparative Analysis}
\todo{Quick intro}
\subsection{Analysis Framework}
\todo{To Cover aspects. e.g. Architectural dimensions, functional capabilities, performance and scalability, usability, extensibility, validation and evaluation}
\subsection{Analysis \& Results}
\todo{Actual Comparison with Results, Strength and Weaknesses, ...}
\subsection{Key Findings \& Highlights}
\todo{Common patterns, performance leaders, comprehensive solutions, specialized tools}
\subsection{Gap Analysis}
\todo{Missing capabilities, limited coverage area, scalability limitations, validation weaknesses, integration challenges, emerging requirements}