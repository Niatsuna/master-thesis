\chapter{State-of-the-Art Simulator Analysis}
A comprehensive analysis of the current edge computing simulation landscape is the focus of the following sections.
The analysis provides the foundation for developing a systematic evaluation framework for edge computing simulators and begins by establishing
clear selection criteria for identifying representative simulators from the extensive body of research and development in this domain.
Following a systematic selection process, ten representative edge computing simulators are examined, each offering distinct approaches to modeling edge computing environments.
The chapter proceeds with detailed simulator profiles that introduce their core architectures, capabilities, and intended use cases, providing readers with essential background for understanding their respective strengths and limitations.
To support the systematic comparison of the ten representative simulators, we establish an analysis framework that defines the aspects and categories in which the comparison should take place.
Subsequently, a structured comparative analysis examines these simulators in comparison to their competitors, revealing significant patterns and variations within the simulation landscape.
These significant key aspects are highlighted, and missing aspects are covered in a comprehensive gap analysis, establishing the foundation for the requirements catalogue developed in the subsequent chapter.

% ------------------------------------------
\section{Selection Criteria and Methodology}
As the edge computing simulation landscape is vast, comparing every existing simulator is neither efficient nor sustainable and does not guarantee that meaningful characteristics are highlighted.
Hence, we select a number of representative simulators to represent the current state-of-the-art for the edge computing simulation landscape.

For this, we must determine which criteria a simulator must meet to be chosen, and the number of simulators must be sufficient to provide a comprehensive view of the landscape.

The first and most critical criterion is that every simulator in this list must be published and accessible as an open-source project, so that a meaningful analysis is possible.
If the simulator is not accessible, no insight can be gained nor can the analysis be reproduced or reviewed. 
Additionally, if the simulator is open-source, the approach by which certain features are implemented can be further reviewed and later referenced for one's own implementation.

Secondly, highly cited or well-established simulators should be present in the list of representatives, as a high citation and reference count demonstrates that the simulators have significant impact in research and development.
Furthermore, if a simulator is highly referenced, it can be considered a foundation upon which other researchers and developers have built, and can therefore demonstrate fundamental capabilities that are important for every simulator.

Lastly, we should include simulators that fulfill a meaningful niche or represent a recent trend, as they show the current development direction in the field.
They also demonstrate the recent adoption of modern technologies in the simulation landscape and provide valuable insight into current challenges faced in edge computing.

Based on these three criteria, we select ten representatives, which we introduce in more detail in the next section.

As a general foundation for our research and analysis, we conduct a literature review of the original paper for each simulator as well as any related papers.
This ensures that we have a general theoretical knowledge base for each simulator, as well as their related components, such as simulators they build upon or reference.
Additionally, we examine the official implementation and documentation to gain broader insight into each feature and the architecture of each simulator.
By testing each simulator, we also ensure that each one is still deployable and works as intended, even if it may be older than others.

% ------------------------------------------
\section{Representative Simulator Overview}
In the following each of the ten representatives gets quickly introduced, by stating which criteria they fulfill and by a quick rundown of their capabilities, foci and current development status as well as their strengths and weaknesses.
This section is only to quickly introduce the representatives and a further more detailed rundown is present in the subsequent section as each important capability gets compared.
% ----------------------------------
\subsection{EdgeCloudSim}
EdgeCloudSim [\cite{sim-edgecloudsim}] is a highly cited and widely adapted edge computing simulator, which focuses on Mobile Edge Computing (MEC) via cloudlets and general edge computing simulation.
With it being first published August 2018, it is one of the older simulators in this list. As of this thesis, EdgeCloudSim was last updated October 2020.

EdgeCloudSim is based on CloudSim [\cite{sim-base-cloudsim}] and presents a modular architecture with advanced mobility modelling in Java.
The general design philospohy is to create a CloudSim-based framework designed for Edge Computing scenarios with focus on both computational and networking resources.
It supports cloud-edge mobile multi-tier topologies and different network types like WLAN, WAN and cellular networks.

EdgeCloudSim is especially designed for MEC scenarios and shows excellent mobility modeling capabilities.
Additionally to that, the mobility modeling gets supported by a realistic network modeling for WLAN and WAN with distance-based latency and due to its modular architecture, EdgeCloudSim allows easy customization.
Sadly, the development halted as of October 2020 and due to it being focused on MEC scenarios the remaining scope is relatively limited.
It is therefore less suitable for pure IoT or cloud computing scenarios and shows limited built-in support for modern artificial intelligence / machine learning workloads.
Furthermore, it is only single-threaded, which leads to scalability limitations, and has no security, privacy or energy modeling.

% ----------------------------------
\subsection{iFogSim2}
iFogSim2 [\cite{sim-ifogsim2}] is another highly cited and widely adapted edge computing simulator, which builds upon its previous version iFogSim [\cite{sim-base-ifogsim1}].
It is, just like EdgeCloudSim, based on CloudSim [\cite{sim-base-cloudsim}] and focuses also on MEC scenarios but extends its scope to IoT, fog computing and microservices.
iFogSim2 is developed as a CloudSim-based framework designed for comprehensive Edge/Fog computing scenarios with focus on mobility, clustering and microservice orchestration, and was first published in September 2021.
The development seems to be relatively inactive, as the last meaningful update was done August 2021.

Just as its competitor EdgeCloudSim, it shows an advanced mobility support but provides ways to include real datasets for more realistic approaches.
As for topology, iFogSim2 allows for multi-tier cloud-fog-edge-devices and custom topologies with their respective supported network and protocol types.

iFogSim2 presents comprehensive fog/edge computing simulation capabilities and advance mobility support with it being able to integrate real datasets.
It has strong academic backing and is best used in IoT and fog computing research as well as academic prototyping and clustering algorithm development.
But iFogSim2 also has its weaknesses:
The scalability is limited by its single-threaded execution and it shows limited real-time simulation capabilities.
Furthermore, the output is console-only and needs manual implementation for file output.
It has a limited built-in security modeling and energy modeling capabilities and no native containerization support.
Therefore, it performs relatively poorly for real-time system simulation requirements, massive parallel processing scenarios, pure cloud computing simulations or 
scenarios requiring extensive energy modeling.

% ----------------------------------
\subsection{YAFS}
Yet Another Fog Simulator (YAFS) [\cite{sim-yafs}] is a highly cited simulator which was first published in July 2019 and last updated June 2022.
Other than iFogSim2 and EdgeCloudSim, YAFS is written in Python 2.7 and extends EdgeSimPy [\cite{sim-base-edgeSimpy}] and focuses on IoT, fog and edge computing as well as resource allocation.
YAFS is designed to be lightweight, robust and highly configurable based on complex network theory with minimal class structure.

It exceeds in being extremly lightweight and being highly configurable and extensible. 
YAFS shows full transparency of simulation data and allows dynamic control of all simulations aspects.
This makes it perfect for research requiring custom algorithms and policies as well as complex network topology analysis and scenarios requiring high customization.

Nonetheless, as YAFS has a relatively inactive development status and is based on a deprecated python version it has some modern flaws.
For example, it has no built-in energy modeling nor security or privacy features.
Just like its competitors, it struggles with being single-threaded and therefore limits its scalability.
With this, YAFS is not ideal for production or commercial environments, large-scale parallel simulations or research that focus on energy or security.

% ----------------------------------
\subsection{Mockfog 2.0}
Mockfog 2.0 [\cite{sim-mockfog2}] is a widely adopted and well established tool in the simulation landscape and is a successor of the previous Mockfog (1.0) [\cite{sim-base-mockfog1}].
It is not a simulator like the previous mentioned, but an cloud-based emulator written in Node.js, which focuses on fog and edge computing for real application testing.
The emulator was first published 2021 and the development halted around June 2021.

Mockfog 2.0 presents a real infrastructure emulation in cloud environments to test fog applications under realistic conditions.
It features docker containerization and automated experiment orchestration while maintaining a real cloud-based emulation with runtime network manipulation.

Due to being a emulation, the resulting behavior is realistic and accurate. 
Additionally, it allows for dynamic,realistic and scalable testing scenarios and reduces manual effort.
With this it succeeds in its main goal of real application testing, but also is useful for validation of prototypes, algorithms and performance.
Additionally, it can be used for infrastructure capacity planning and industrial research making it quite adaptable.

Nevertheless, due to having a cloud infrastructure usage it has high cost and a complex setup.
It is not suitable for large-scale theoretical studies and cannot test theoretical algorithms easily.
Additionally, it has limited mobility support making it inefficient for some projects.

% ----------------------------------
\subsection{EmuFog}
EmuFog [\cite{sim-emufog}] is highly citated, widely adopted and well established extensible emulation framework for large-scale fog computing written in Kotlin and based on MaxiNet\footnote{\url{https://maxinet.github.io/}} [\cite{sim-base-maxinet}] and Mininet\footnote{\url{https://mininet.org/}} [\cite{sim-base-mininet}].
\todo{Is it okay to have a citation and a footnote ? is it too much ? Idk}
First published in October 2017 and updated until September 2020, EmuFog focuses on large-scale fog computing infrastructure emulation while mainly addressing the network emulation.

It provides highly accurate results and supports docker containerization and both synthetic and real-world topologies.
EmuFog has a good scalable architecture and efficient fog node placement algorithms with cost optimizations, making it a sufficient tool for network performance analysis 
in fog environments, application deployment and orchestration testing as well as large-scale fog infrastructure evaluation.

On the other hand, EmuFog has no mobility nor energy modeling and requires high computational resources for large-scale emulation.
Due to that, large projects are expensive and mobile or energy projects or research is not recommended with EmuFog.
Additionally, its overhead is quite high, making it inefficient for quick prototyping and algorithm development.

% ----------------------------------
\subsection{EdgeAISim}
EdgeAISim [\cite{sim-edgeAIsim}] is a relatively new simulator, first published in October 2023, and focuses on the growing trend of artificial intelligence on the edge.
It features an extension to EdgeSimPy [\cite{sim-base-edgeSimpy}], which is written in Python, and focuses on resource management and energy optimization for artificial intelligence and machine learning workflows on the edge.

The integration of advanced AI models for intelligent resource management as well as the use of advanced artificial intelligence techniques like multi-armed bandit, Deep Q-Networks (DQN) and GNNs, makes it an 
efficient simulator in regards to scenarios of energy optimization studies, intelligent resource management and task scheduling algorithm development on the edge.

As EdgeAISim is a relatively small extension and has a very strict focus, it lacks in multiple other features that its competitors present.
Due to this, scenarios without artificial intelligence requirements don't reap any benefits from this simulator, as the workflow either hindered by additional workload or is simply 
delegated to EdgeSimPy.

% ----------------------------------
\subsection{FogNetSim++}
FogNetSim++ [\cite{sim-fognetsim++}] is relatively well known and established tool written in C++ and first published October 2018.
It is based on the OMNeT++\footnote{\url{https://omnetpp.org/}} and INET\footnote{\url{https://inet.omnetpp.org/}} Framework and focuses on network-centric fog environemnts with distributed fog systems.
The development halted as of the same month.

FogNetSim++ succeeds in advance network modeling with packet-level precision and with the built on proven OMNeT++ framework it shows extensive networking capabilties.
Furthermore, it present excellent scalabilities for large-scale network simulations and broad selection of debunnging and visualization tools.
Additionally, it also present deterministic and reproducible simulation results.
This supports network performance analysis and protocol development as well as large-scale network simulation with fog nodes and academic research that requires high network simulation accuracy.

However, due to being highly reliant on OMNeT++ and INET, further development and customization of the simulator require expertise and shows a steep learning curve and complex setup.
Adittionally, the documentation is relatively limited and due to the focus primarly laying on networking other components are being neglected.
This makes the simulator relatively inefficient for quick prototyping and real-time system simulation as well as energy- and security-focused studies.

% ----------------------------------
\subsection{FogTorch$\Pi$}
FogTorch$\Pi$ [\cite{sim-fogtorchpi}] is a well established probabilistic analysis tool in graph structure application deployment.
Written in Java in May 2017, FogTorch$\Pi$ focuses on fog computing, Quality of Service (QoS) aware deployment optimization as well as IoT application deployment.
With its probabilistic QoS-assurance and resource consumption estimation for eligible deployments of fog application with focus on deployment descision support, it serves a unique focus in the simulation landscape.
No visible development can be observed since January 2019.

It succeeds in clear separation between infrastructure and application modeling as well as cost-aware deployment optimization.
Therefore, it is best used for fog application deployment planning, QoS-aware deployment optimization, cost analysis for fog deployments and general academic research on deployment strategies in the fog computing field.

However, it features no runtime simulation capabilties and the deployment analysis is rather static with no dynamic reconfiguration or adaptation.
Furthermore, no ernergy or security modeling is present and its limited scalability makes it inefficient for energy or security based large-scaled projects.
% ----------------------------------
\subsection{Fogify}
Fogify [\cite{sim-fogify}] is a Python based emulator that focuses on fog/edge computing in a cloud-native container orchestration based network emulation.
It is designed to be realistic, reproducible and to server scalable fog/edge testbeds using containers.
First published in November 2020, Fogify shows a relatively active development with a modern docker and kubernetes based approach for linux networking.

Due to being realistic and reproducible fog/edge testbeds, it shines at testing real applications in such scenarios.
Furthermore, due to its container and kubernetes integration it supports declarative scenario description and therefore being relatively easy to configure and quite modern.

Sadly, it features no built-in energy modeling which makes it insufficient for energy-focused research unless the energy-consumption is instrumented externally.
Large-scale and time-accelerated scenarios also struggle with Fogify, as its hardware-limited scalability stands in the way.

% ----------------------------------
\subsection{iContinuum}
iContinuum [\cite{sim-icontinuum}] joins Fogify as a python based emulator that uses docker containerization and kubernetes to emulate linux networking.
Here, it is also designed to be realistic and reproducible, but other than the previous, focuses on cloud-to-edge, orchestration and reproducibility of large-scale experimentations. 

iContinuum succeeds in its goals by serving realistic and reproducible cloud-to-edge continuum testbeds while maintaining a multi-domain orchestration and declarative scenario description.
It is therefore suitable for orchestration research across the cloud-edge continuum, testing real applications in distributed scenarios and in network-based resource emulation.

However, just like its competitor, it is hardware-limited in its scalability and has no synthetic time like simulators in this field.
Additionally, it has no built-in energy modeling, making it not as suitable for large-scale, time-accelerated or energy-focused simulation.

% ------------------------------------------
\section{Comparative Analysis}
\todo{Quick intro}
% ----------------------------------
\subsection{Analysis Framework}
\todo{To Cover aspects. e.g. Architectural dimensions, functional capabilities, performance and scalability, usability, extensibility, validation and evaluation}
% ----------------------------------
\subsection{Analysis \& Results}
\todo{Actual Comparison with Results, Strength and Weaknesses, ...}
% ----------------------------------
\subsection{Key Findings \& Highlights}
\todo{Common patterns, performance leaders, comprehensive solutions, specialized tools}
% ----------------------------------
\subsection{Gap Analysis}
\todo{Missing capabilities, limited coverage area, scalability limitations, validation weaknesses, integration challenges, emerging requirements}