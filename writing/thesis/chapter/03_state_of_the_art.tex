\chapter{State-of-the-Art Simulator Analysis}
A comprehensive analysis of the current edge computing simulation landscape is the focus of the following sections.
The analysis provides the foundation for developing a systematic evaluation framework for edge computing simulators and begins by establishing
clear selection criteria for identifying representative simulators from the extensive body of research and development in this domain.
Following a systematic selection process, ten representative edge computing simulators are examined, each offering distinct approaches to modeling edge computing environments.
The chapter proceeds with detailed simulator profiles that introduce their core architectures, capabilities, and intended use cases, providing readers with essential background for understanding their respective strengths and limitations.
To support the systematic comparison of the ten representative simulators, we establish an analysis framework that defines the aspects and categories in which the comparison should take place.
Subsequently, a structured comparative analysis examines these simulators in comparison to their competitors, revealing significant patterns and variations within the simulation landscape.
These significant key aspects are highlighted, and missing aspects are covered in a comprehensive gap analysis, establishing the foundation for the requirements catalogue developed in the subsequent chapter.

% ------------------------------------------
\section{Selection Criteria and Methodology}
As the edge computing simulation landscape is vast, comparing every existing simulator is neither efficient nor sustainable and does not guarantee that meaningful characteristics are highlighted.
Hence, we select a number of representative simulators to represent the current state-of-the-art for the edge computing simulation landscape.

For this, we must determine which criteria a simulator must meet to be chosen, and the number of simulators must be sufficient to provide a comprehensive view of the landscape.

The first and most critical criterion is that every simulator in this list must be published and accessible as an open-source project, so that a meaningful analysis is possible.
If the simulator is not accessible, no insight can be gained nor can the analysis be reproduced or reviewed. 
Additionally, if the simulator is open-source, the approach by which certain features are implemented can be further reviewed and later referenced for one's own implementation.

Secondly, highly cited or well-established simulators should be present in the list of representatives, as a high citation and reference count demonstrates that the simulators have significant impact in research and development.
Furthermore, if a simulator is highly referenced, it can be considered a foundation upon which other researchers and developers have built, and can therefore demonstrate fundamental capabilities that are important for every simulator.

Lastly, we should include simulators that fulfill a meaningful niche or represent a recent trend, as they show the current development direction in the field.
They also demonstrate the recent adoption of modern technologies in the simulation landscape and provide valuable insight into current challenges faced in edge computing.

Based on these three criteria, we select ten representatives, which we introduce in more detail in the next section.

As a general foundation for our research and analysis, we conduct a literature review of the original paper for each simulator as well as any related papers.
This ensures that we have a general theoretical knowledge base for each simulator, as well as their related components, such as simulators they build upon or reference.
Additionally, we examine the official implementation and documentation to gain broader insight into each feature and the architecture of each simulator.
By testing each simulator, we also ensure that each one is still deployable and works as intended, even if it may be older than others.

% ------------------------------------------
\section{Representative Simulator Overview}
In the following each of the ten representatives gets quickly introduced, by stating which criteria they fulfill and by a quick rundown of their capabilities, foci and current development status as well as their strengths and weaknesses.
This section is only to quickly introduce the representatives and a further more detailed rundown is present in the subsequent section.
% ----------------------------------
\subsection{EdgeCloudSim}
EdgeCloudSim [\cite{sim-edgecloudsim}] is a highly cited and widely adapted edge computing simulator, which focuses on Mobile Edge Computing (MEC) via cloudlets and general edge computing simulation.
With it being first published August 2018, it is one of the older simulators in this list. As of this thesis, EdgeCloudSim was last updated October 2020.

EdgeCloudSim is based on CloudSim [\cite{sim-base-cloudsim}] and presents a modular architecture with advanced mobility modelling in Java.
The general design philospohy is to create a CloudSim-based framework designed for Edge Computing scenarios with focus on both computational and networking resources.
It supports cloud-edge mobile multi-tier topologies and different network types like WiFi, WAN and cellular networks.

EdgeCloudSim is especially designed for MEC scenarios and shows excellent mobility modeling capabilities.
Additionally to that, the mobility modeling gets supported by a realistic network modeling for WiFi and WAN with distance-based latency and due to its modular architecture, EdgeCloudSim allows easy customization.
Sadly, the development halted as of October 2020 and due to it being focused on MEC scenarios the remaining scope is relatively limited.
It is therefore less suitable for pure IoT or cloud computing scenarios and shows limited built-in support for modern artificial intelligence / machine learning workloads.
Furthermore, it is only single-threaded, which leads to scalability limitations, and has no security, privacy or energy modeling.

% ----------------------------------
\subsection{iFogSim2}
iFogSim2 [\cite{sim-ifogsim2}] is another highly cited and widely adapted edge computing simulator, which builds upon its previous version iFogSim [\cite{sim-base-ifogsim1}].
It is, just like EdgeCloudSim, based on CloudSim [\cite{sim-base-cloudsim}] and focuses also on MEC scenarios but extends its scope to IoT, fog computing and microservices.
iFogSim2 is developed as a CloudSim-based framework designed for comprehensive Edge/Fog computing scenarios with focus on mobility, clustering and microservice orchestration, and was first published in September 2021.
The development seems to be relatively inactive, as the last meaningful update was done August 2021.

Just as its competitor EdgeCloudSim, it shows an advanced mobility support but provides ways to include real datasets for more realistic approaches.
As for topology, iFogSim2 allows for multi-tier cloud-fog-edge-devices and custom topologies with their respective supported network and protocol types.

iFogSim2 presents comprehensive fog/edge computing simulation capabilities and advance mobility support with it being able to integrate real datasets.
It has strong academic backing and is best used in IoT and fog computing research as well as academic prototyping and clustering algorithm development.
But iFogSim2 also has its weaknesses:
The scalability is limited by its single-threaded execution and it shows limited real-time simulation capabilities.
Furthermore, the output is console-only and needs manual implementation for file output.
It has a limited built-in security modeling and energy modeling capabilities and no native containerization support.
Therefore, it performs relatively poorly for real-time system simulation requirements, massive parallel processing scenarios, pure cloud computing simulations or 
scenarios requiring extensive energy modeling.

% ----------------------------------
\subsection{YAFS}
Yet Another Fog Simulator (YAFS) [\cite{sim-yafs}] is a highly cited simulator which was first published in July 2019 and last updated June 2022.
Other than iFogSim2 and EdgeCloudSim, YAFS is written in Python 2.7 and extends EdgeSimPy [\cite{sim-base-edgeSimpy}] and focuses on IoT, fog and edge computing as well as resource allocation.
YAFS is designed to be lightweight, robust and highly configurable based on complex network theory with minimal class structure.

It exceeds in being extremly lightweight and being highly configurable and extensible. 
YAFS shows full transparency of simulation data and allows dynamic control of all simulations aspects.
This makes it perfect for research requiring custom algorithms and policies as well as complex network topology analysis and scenarios requiring high customization.

Nonetheless, as YAFS has a relatively inactive development status and is based on a deprecated python version it has some modern flaws.
For example, it has no built-in energy modeling nor security or privacy features.
Just like its competitors, it struggles with being single-threaded and therefore limits its scalability.
With this, YAFS is not ideal for production or commercial environments, large-scale parallel simulations or research that focus on energy or security.

% ----------------------------------
\subsection{Mockfog 2.0}
% ----------------------------------
\subsection{EmuFog}
% ----------------------------------
\subsection{EdgeAISim}
% ----------------------------------
\subsection{FogNetSim++}
% ----------------------------------
\subsection{FogTorch$\Pi$}
% ----------------------------------
\subsection{Fogify}
% ----------------------------------
\subsection{iContinuum}

% ------------------------------------------
\section{Comparative Analysis}
\todo{Quick intro}
% ----------------------------------
\subsection{Analysis Framework}
\todo{To Cover aspects. e.g. Architectural dimensions, functional capabilities, performance and scalability, usability, extensibility, validation and evaluation}
% ----------------------------------
\subsection{Analysis \& Results}
\todo{Actual Comparison with Results, Strength and Weaknesses, ...}
% ----------------------------------
\subsection{Key Findings \& Highlights}
\todo{Common patterns, performance leaders, comprehensive solutions, specialized tools}
% ----------------------------------
\subsection{Gap Analysis}
\todo{Missing capabilities, limited coverage area, scalability limitations, validation weaknesses, integration challenges, emerging requirements}