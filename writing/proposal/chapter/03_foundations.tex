\chapter{Foundations and Technologies}
In following sections present an overview of foundations and technologies, which are viable for the thesis.
This chapter can be seen as a starting point and will be visited in more detail in the thesis.

\section{Edge Computing}
Edge Computing represents a distributed computing pattern that brings computation and data storage closer to the source of data generation and
therefore reduces latency bandwidth usage [\cite{7488250}]. In comparison the traditional approach of cloud computing, which centralizes resources in distant
data centers, edge computing distributes computational resources across geographically dispersed nodes positioned at the so called edge of the network.
\\
Some commonly known projects for edge computing are self-driving cars, that uses sensor data and make split-second decisions [\cite{9546662}], 
smart glasses, that read real-time visual data like gesture recognition and spatial tracking to determine actions [\cite{10134901}], and to some degree smart homes, that access local sensor data like from thermostats, security cameras and voice recognition and process these locally for more routine commands [\cite{7488250}].
The latter also uses cloud computing for more complex commands.

\section{Kubernetes}
Kubernetes is an open-source container orchestration platform that automates the deployment, scaling and management of containerized applications across distributed computing environments by implementing a master-worker architecture [\cite{10.5555/3175917}].
It provides a robust framework for managing microservices architectures by abstracting underlying infrastructe complexity.
Core abstractions include Pods, as the smallest deployable units, Services for network endpoints and deployments for lifecycle management.

Based on the high abstraction, lifecycle management and robustness, Kubernetes became a highly respected and mostly standard framework in the industry and research field [\cite{10.1145/3539606}].

\section{Ecoscape}
\todo{1) Write the subsections 2) Do i need to cite Ecoscape here? If yes: How ?}
Ecoscape allows for configurations of scenarios that run on a Kubernetes environment and provides means to evaluate the performance of specific operators within this environment.
It focus highly on fault tolerance, scalability and the optimization of middleware.
To achieve this, multiple technologies beside Kubernetes are used.
\subsection{Prometheus}
\subsection{Kafka}
\subsection{Kepler}
\subsection{Chaos Mesh}
