\chapter{Introduction}
% -----------------------------------------------------------------------------
\section{Motivation}
  With IoT devices projected to grow [\cite{7488250}, \cite{10258346}], the demand for real-time, low-latency applications has led to the development of edge computing as a complement and extension for cloud computing [\cite{ASHOURI2021100346}]
  Edge computing brings computational resources closer to data sources and end users, reducing latency, minimizing bandwidth usage and enabling applications that require immediate response times.

  However, deploying and testing edge computing solutions in real-world scenarios presents significant challenges, that consists of but not only of physical constraints like high costs, time commitments but are also very impractical for research and development purposes [\cite{ASHOURI2021100346}, \cite{7488250}].
  This has led to the development of various edge computing simulators that aim to model and simulate the behavior, performance and characteristics of distributed edge environments.
  These simulators are essential tools to analyze resource allocations, algorithms, configurations and deployment strategies before committing to a physical implementation.
  As containerized applications and orchestration platforms like Kubernetes become the standard for researchers and developers alike, the need to extend these technologies to the edge environments has become critical.


  Despite the growing number of available edge computing simulators, they all show different aspects, potentially incomplete or biased simulation environments [\cite{ASHOURI2021100346}].
  This lack of unison makes it difficult for researcher and developers to select appropriate simulation platforms and limits the development of more effective edge computing solutions.
  Particularly in the context of Kubernetes-based edge deployments, which have become increasingly prevalent, there is a gap in understanding which simulation tools best capture the complexities of container orchestration on the edge.


  Furthermore, existing simulators could focus on specific aspects of edge computing while potentially overlooking other critical factors that incluence real-world performance.
  Without a clear understanding of what constitutes a comprehensive edge computing simulator, research and development of newer simulators can become difficult and risky.

  
  The planned master's thesis therefore presents the advancing of the edge computing simulator ecoscape based on an evaluation of the edge computing simulation landscape.

% -----------------------------------------------------------------------------
\section{Document Structure}
To further explain the intention of said master's thesis, Chapter 2 introduces the main goal the master's thesis should meet.
Chapter 3 presents the necessary foundations for the presented topic, followed by the planned steps for the goal in Chapter 4.
At the end, Chapter 5 presents the currently planned schedule.